\chapter{Investigations for testnet deployment}
\label{chap:chapterfive}

\section{Testnet Installation process}

\subsection{icos\_deploy.sh}

\begin{enumerate}
    \item Uses the \texttt{inventory.py} to generate the dynamic mapping of nodes to their IPv6 addresses.
    \item Creates USB sticks for IC Nodes.
\end{enumerate}

To interact with the NNS on <testnet>, on which the NNS canisters have already been deployed:
\begin{itemize}
    \item Add the \texttt{network} in \texttt{dfx.json}. Point at a machine from the NNS subnet from \texttt{env/<testnet>/
    hosts}. It's already done for the NNS testnet.
    \item Depending on the type of "network", \texttt{dfx} will expect the mapping between canister names and IDs to either be 
    in \texttt{./canister\_ids.json} (the current directory must be the NNS dir) or \texttt{./.dfx/<networkname>/canister\_ids.
    json}.
    \item Add entries for each \texttt{canister} in \texttt{canister\_ids.json}: duplicate the existing one and just change the 
    network name. The canister IDs are identical for all subnets.
    \item For each canister, copy its \texttt{.did} file under 
    \texttt{.dfx/<networkname>/canisters/<canister\_name>/  <canister\_name>.did}.
    \item Creates USB sticks for boundary nodes.
    \item Runs the \texttt{ic\_network\_redeploy} playbook with \texttt{ic\_state=create}.
    \item This playbook gives the role of \texttt{ic\_guest}.
\end{itemize}

\subsection{ic\_guest}

The \texttt{main.yml} file consists of the following tasks:

\begin{itemize}
    \item \texttt{prepare.yml}: It sets the output of the commands, creates some folders, and installs \texttt{GNU parallel} and 
    \texttt{zstd}.
    \item \texttt{disk\_pull.yml}: This task downloads disk images for different node types from a specified URL and unarchives 
    them.
    \item \texttt{disk\_push.yml}: This task archives the \texttt{disk.img} file into a \texttt{disk-img.tar.zst} file and 
    synchronizes it to the remote host(s).
    \item \texttt{aux\_disk\_push.yml}: This task performs the same operations as \texttt{disk\_push.yml}, but for the auxiliary 
    disk.
    \item \texttt{media\_pull.yml}: Debug message for CI/CD Pipelines.
    \item \texttt{media\_push.yml}: This task copies the media image file (\texttt{media.img}) to the remote hosts.
    \item \texttt{create.yml}: This task copies various files and prepares the guest template file, then defines and starts a guest virtual machine.
\end{itemize}

\subsection{icos\_network\_redeploy.yml}

\begin{itemize}
    \item This playbook is started with the \texttt{ic\_state=start} parameter and has the role of \texttt{ic\_guest}.
    \item It starts the virtual machine guest and sets it to autostart using the \texttt{virsh} command.
\end{itemize}

\subsection{icos\_network\_redeploy.yml}

\begin{itemize}
    \item This playbook is started with the \texttt{ic\_state=install} parameter and has the role of \texttt{ic\_guest}.
    \item It installs the NNS canisters and verifies the installation by checking the subnet.
    \item The tasks are executed on the localhost using the \texttt{delegate\_to: localhost} directive.
\end{itemize}

\section{NNS Installation process}

\subsection{Wait for replica to listen on all NNS nodes on port 8080}

This task waits for the application to start listening on port 8080 for all nodes defined in the 'nns' group.

\subsection{Check if the initial neuron config exists}

This task checks if the \texttt{initial-neurons.csv} file exists.

\subsection{Get Custom NNS canisters}

This task copies the custom wasm file into the \texttt{canister} folder.

\subsection{Install NNS Canisters}

This task installs and verifies the installation of the NNS (Neuron Network Service) canisters using shell commands. It 
initializes the NNS canisters on a subnet and checks the installation.

The task is executed on the localhost.

\section{IC-OS Installation process}

\begin{enumerate}
    \item Install \texttt{virsh}.
    \item Install the required packages: \texttt{libvirt-daemon-system}, \texttt{libvirt-clients}, \texttt{bridge-utils}, \texttt
    {virtinst}, and \texttt{virt-manager}.
    \item Add the user to the \texttt{kvm} and \texttt{libvirt} groups.
    \item Download the disk image and ISO image for the virtual machine.
    \item Start the default network using \texttt{virsh net-start default}.
    \item Start the virtual machine using the \texttt{virt-install} command with appropriate parameters.
\end{enumerate}

\section{Our findings}

\textbf{Main entrypoint}

The main entrypoint is the \texttt{testnet/tools/icos\_deploy.sh} script.

\textbf{Steps:}

1. \textbf{Pre-requisites:}
    \begin{enumerate}
        \item Install the required packages:
        \texttt{apt -y install ansible coreutils jq mtools rclone tar util-linux unzip --no-install-recommends}
        \item Install \texttt{mkfs}:
        \texttt{apt-get install dosfstools}
        \item Get the SHA of the disk image:
        \texttt{./gitlab-ci/src/artifacts/newest\_sha\_with\_disk\_image.sh origin/master}
        \item Start the deployment:
        \texttt{./testnet/tools/icos\_deploy.sh <testnet> --git-revision d53b551dc677a82c8420a939b5fee2d38f6f1e8b}
    \end{enumerate}

2. \textbf{Installation process}
    
    \begin{enumerate}
        \item \textbf{Displays local IPv4 and IPv6 address info}
        \item \textbf{Destroy the previous deployments}
        \item \textbf{Build USB sticks for IC nodes}
        
        This step downloads the canisters, release files, etc. Initially, it tries to get them from the S3 bucket, but then it 
        falls back to other options such as using \texttt{rclone} to download from somewhere else.
        
        \item \textbf{Build USB sticks for boundary nodes}
        
        There seems to be an error related to undefined variables.
        
        \item \textbf{Run \texttt{icos\_redeploy.yml} with \texttt{create} state}
        
        This playbook is used to deploy the ICOS to the nodes. The inventory file (\texttt{inventory.py}) is used to manage the 
        nodes and define the groups. The IPv6 addresses are assigned using the shared config and host names defined in \texttt
        {hosts.ini}.
        
        \item \textbf{Run the playbook \texttt{icos\_network\_redeploy.yml}}
        
        However, there are difficulties in updating the IPs as they are generated dynamically, causing the playbook to crash 
        when trying to SSH into the nodes.
    \end{enumerate}


\section{Inferences and Learnings}

\begin{itemize}
\item They directly initialize the NNS from the source code. From the source code, they get the WebAssembly (wasm) for the other 
canister.
\item There is a strong dependency on the IPV6 address in order to complete the installation.
\item Boundary nodes use host- and guest-OS architecture, which is based on standard Ubuntu and often referred to as "IC-OS." 
This is what is being pulled from their server and being installed and run on these nodes. This is something that we would need 
to run on our nodes as well.
\end{itemize}
