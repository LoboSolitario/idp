Shouvik GhoshThe first sentence of an abstract should clearly introduce the topic of the paper so that readers can relate it to other work they are familiar with.
However, an analysis of abstracts across a range of fields show that few follow this advice, nor do they take the opportunity to summarize previous work in their second sentence.
A central issue is the lack of structure in standard advice on abstract writing, so most authors don’t realize the third sentence should point out the deficiencies of this existing research.
To solve this problem, we describe a technique that structures the entire abstract around a set of six sentences, each of which has a specific role, so that by the end of the first four sentences you have introduced the idea fully.
This structure then allows you to use the fifth sentence to elaborate a little on the research, explain how it works, and talk about the various ways that you have applied it, for example to teach generations of new graduate students how to write clearly.
This technique is helpful because it clarifies your thinking and leads to a final sentence that summarizes why your research matters.

\textit{The text above describes the contents and structure of an abstract.
The text by Steve Easterbrook~\cite{abstractwriting} presents the structure of typical abstract for a scientific paper.
The abstracts of theses are typically not as space-constraint as scientific papers, therefore, abstracts tend to get longer.}
