\chapter{Introduction}
\label{chap:chaptertwo}


With the growing adoption of blockchain technology, the number of readily-available solutions has multiplied dramatically. 
Approximately five thousand distinct cryptocurrencies have been reported on a single website[1]. There are a growing 
number of Layer-1 blockchains currently, and each of these implementations aims at offering improvements through distinctive features 
focused on the performance and application to various use cases. The authors of different blockchain protocols claim impressive 
performance. These results are usually obtained in isolation and are often non-reproducible, which makes them hard to compare and 
verify. Hence it is crucial to run these experiments on a reproducible architecture in order to truly understand the performance.

Blockchain technology, since its inception, has grown exponentially, with various chains being developed, each offering a unique 
approach to consensus, security, and performance. Dfinity is particularly an intriguing case as it has a fundamentally very different 
architecture from fundamental blockchains. The IC consists of a set of cryptographic protocols that connect independently operated 
nodes into a collection of blockchains and hence allowing smart contracts to implement fully decentralized applications hosted 
completely on the blockchain.

\section{Problem Statement}

Internet Computer (IC), developed by the Dfinity Foundation, with its goal of creating ``blockchain singularity", provides a unique 
proposition in the expansive world of blockchain technology, making it interesting for developers and researchers. Dfinity 
foundation, the core developers of IC, claim 250,000+ Ethereum equivalent transactions per second and 37.3 MB/s block throughput 
capacity \cite{dfinitywebsite}, which is a game changer in the field. However, there is no clarity/reproducability for the 
benchmarked data. There is no way to understand the conditions under which these benchmarking was done. Therefore, for this project, 
we focus on an in-depth exploration and analysis of the Internet Computer blockchain from the perspective of benchmarking it.

\section{Goals and Research Questions}

Based on the problem statement, we define two research goals and the questions that help us to achieve the goals

\textbf{Goal 1:}  In-depth exploration and analysis of the Internet Computer blockchain in order to run a local testnet.

\textbf{Question 1:} How does the Internet Computer blockchain differ from other blockchain models in terms of architecture and 
functionality?

\textbf{Question 2:} How can the \textbf{`dfx'} SDK from Dfinity be used to set up and manage replicas on a local nodes and can we 
enable communication between multi-node setup?

\textbf{Question 3:} What are the specific requirements to set up a local Internet Computer blockchain testnet?

\textbf{Goal 2:} Explore the benchmarking possibilities, experiment types, and performance evaluations within a single \textbf{`dfx'} 
environment, on the mainnet, and by connecting a local node from a testbed to the mainnet.

\textbf{Question 1:} what can be benchmarked using a single dfx environment?

\textbf{Question 2:} What types of experiments can we do directly on the mainnet?

\textbf{Question 3:} If we can run a node on our current testbed and connect it to the mainnet, what benchmarking can be performed?

\section{Methodology and Structure}

To understand IC performance, we aimed to deploy it on our local testbed and conduct a series of benchmarking experiments. First, we 
looked into `dfx', the IC SDK, in order to set up local replicas in multiple nodes and enable communication between them. Secondly, 
we wanted to have a complete local testnet setup. However, unlike other layer 1 solutions, Dfinity does not readily support local 
testnet setups, creating an initial roadblock for our research.

Therefore, the first portion of our project pivoted towards understanding Dfinity's complex infrastructure in its entirety. This 
necessitated a deep dive into its technical architecture and learning the intricacies of the process required to install a local 
version of the Dfinity testnet.

Following this, our objective has been to leverage the insights gained from this initial research to deploy the Dfinity blockchain 
locally and subsequently understand how to execute benchmarking experiments (i.e. figuring out the workloads, understanding the 
important metrics etc).

This report details our understanding of the Dfinity blockchain, the challenges we faced and overcame in deploying a local testnet, 
and what are the future possibilities. Our findings will offer valuable insights into the performance of the Dfinity blockchain and 
further contribute to the body of knowledge in this rapidly evolving field.
