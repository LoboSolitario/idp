\documentclass[NET,english,beameralt]{tumbeamer}

% If you load additional packages, do so in packages.sty as figures are build
% as standalone documents and you may want to have effect on them, too.

% Folder structure:
% .
% ├── beamermods.sty                  % depricated an will be removed soon
% ├── compile                         % remotely compile slides
% ├── figures                         % all figures go here
% │   └── schichtenmodelle_osi.tikz   % each .tikz or .tex is a target
% ├── include                         % create your document here
% │   ├── example.tex                 % example document
% │   └── slides.tex                  % make document wide changes here
% ├── lit.bib                         % literature
% ├── Makefile
% ├── moeptikz.sty                    % fancy networking symbols
% ├── packages.sty                    % load additional packages there
% ├── pics                            % binary pcitures go here
% ├── slides.tex                      % main document (may be more than one)
% ├── tumbeamer.cls
% ├── tumcolor.sty                    % TUM color definitions
% ├── tumcontact.sty                  % TUM headers and footers
% ├── tumlang.sty                     % TUM names and language settings
% └── tumlogo.sty                     % TUM logos

% Configure author, title, etc. here:
\documentclass[NET,english,beameralt]{tumbeamer}

% If you load additional packages, do so in packages.sty as figures are build
% as standalone documents and you may want to have effect on them, too.

% Folder structure:
% .
% ├── beamermods.sty                  % depricated an will be removed soon
% ├── compile                         % remotely compile slides
% ├── figures                         % all figures go here
% │   └── schichtenmodelle_osi.tikz   % each .tikz or .tex is a target
% ├── include                         % create your document here
% │   ├── example.tex                 % example document
% │   └── slides.tex                  % make document wide changes here
% ├── lit.bib                         % literature
% ├── Makefile
% ├── moeptikz.sty                    % fancy networking symbols
% ├── packages.sty                    % load additional packages there
% ├── pics                            % binary pcitures go here
% ├── slides.tex                      % main document (may be more than one)
% ├── tumbeamer.cls
% ├── tumcolor.sty                    % TUM color definitions
% ├── tumcontact.sty                  % TUM headers and footers
% ├── tumlang.sty                     % TUM names and language settings
% └── tumlogo.sty                     % TUM logos

% Configure author, title, etc. here:
\documentclass[NET,english,beameralt]{tumbeamer}

% If you load additional packages, do so in packages.sty as figures are build
% as standalone documents and you may want to have effect on them, too.

% Folder structure:
% .
% ├── beamermods.sty                  % depricated an will be removed soon
% ├── compile                         % remotely compile slides
% ├── figures                         % all figures go here
% │   └── schichtenmodelle_osi.tikz   % each .tikz or .tex is a target
% ├── include                         % create your document here
% │   ├── example.tex                 % example document
% │   └── slides.tex                  % make document wide changes here
% ├── lit.bib                         % literature
% ├── Makefile
% ├── moeptikz.sty                    % fancy networking symbols
% ├── packages.sty                    % load additional packages there
% ├── pics                            % binary pcitures go here
% ├── slides.tex                      % main document (may be more than one)
% ├── tumbeamer.cls
% ├── tumcolor.sty                    % TUM color definitions
% ├── tumcontact.sty                  % TUM headers and footers
% ├── tumlang.sty                     % TUM names and language settings
% └── tumlogo.sty                     % TUM logos

% Configure author, title, etc. here:
\documentclass[NET,english,beameralt]{tumbeamer}

% If you load additional packages, do so in packages.sty as figures are build
% as standalone documents and you may want to have effect on them, too.

% Folder structure:
% .
% ├── beamermods.sty                  % depricated an will be removed soon
% ├── compile                         % remotely compile slides
% ├── figures                         % all figures go here
% │   └── schichtenmodelle_osi.tikz   % each .tikz or .tex is a target
% ├── include                         % create your document here
% │   ├── example.tex                 % example document
% │   └── slides.tex                  % make document wide changes here
% ├── lit.bib                         % literature
% ├── Makefile
% ├── moeptikz.sty                    % fancy networking symbols
% ├── packages.sty                    % load additional packages there
% ├── pics                            % binary pcitures go here
% ├── slides.tex                      % main document (may be more than one)
% ├── tumbeamer.cls
% ├── tumcolor.sty                    % TUM color definitions
% ├── tumcontact.sty                  % TUM headers and footers
% ├── tumlang.sty                     % TUM names and language settings
% └── tumlogo.sty                     % TUM logos

% Configure author, title, etc. here:
\input{include/slides}

\begin{document}

% If you are preparing a talk but do not like the default font sizes, you may
% want to try the class option 'beameralt', which uses smaller default font
% sizes and integrates subsection/subsubsection names into the headline.

% For lecture mode, you may want to build one set of slides per chapter but
% with common page numbering. If so,
% 1) create a new .tex file for each chapter, e.g. slides_chapN.tex,
% 2) set the part counter to N-1 (assuming chapters start at 0), and
% 3) and name your chapter by using the \part{} command.
%\setcounter{part}{-1}
%\part{Organisatorisches und Einleitung}

% For 16:9 slides, use the class option 'aspectratio=169'.

% If class option 'noframenumbers' is given, frame numbers are not printed.

% If class option 'notitleframe' is given, the title frame is not autmatically
% generated.

% Class option 'nocontentframes' suppresses automatic generation of content
% frames when new parts/sections are started.

% Include source files from ./include (or ./include/chapN).
\input{include/example}

% Include markdown source from ./pandoc
%\input{pandoc/example}

% Comment out if you do not want a bibliography
\section{Bibliography}
\begin{frame}[allowframebreaks]
    \bibliographystyle{abbrv}
    \setbeamertemplate{bibliography item}[text]
    \footnotesize
    \bibliography{lit}
\end{frame}

\end{document}



\begin{document}

% If you are preparing a talk but do not like the default font sizes, you may
% want to try the class option 'beameralt', which uses smaller default font
% sizes and integrates subsection/subsubsection names into the headline.

% For lecture mode, you may want to build one set of slides per chapter but
% with common page numbering. If so,
% 1) create a new .tex file for each chapter, e.g. slides_chapN.tex,
% 2) set the part counter to N-1 (assuming chapters start at 0), and
% 3) and name your chapter by using the \part{} command.
%\setcounter{part}{-1}
%\part{Organisatorisches und Einleitung}

% For 16:9 slides, use the class option 'aspectratio=169'.

% If class option 'noframenumbers' is given, frame numbers are not printed.

% If class option 'notitleframe' is given, the title frame is not autmatically
% generated.

% Class option 'nocontentframes' suppresses automatic generation of content
% frames when new parts/sections are started.

% Include source files from ./include (or ./include/chapN).
\section{Section heading}

\begin{frame}
    \frametitle{Example frame}
    \begin{itemize}
        \item item 1
        \item $\ldots$
        \begin{itemize}
            \item test
            \item $\ldots$
        \end{itemize}
    \end{itemize}
    Citation \cite{rfc959}\footnote{A footnote}

    \paragraph{Math mode should be fully functional:}
    $$
    \hat s
    \overline s
    \mathcal S
    \mathbit S
    \mathbit \Lambda
    \sum
    \pd{\xi}
    \pr{X=0}
    \mathbit 1
    $$
\end{frame}

\begin{frame}
    \frametitle{Figures}
    \begin{figure}
        \centering
        \includegraphics[width=.5\textwidth]{figures/example}
        \caption{Figure caption}
        \label{Maizaso0}
    \end{figure}
    Figure~\ref{Maizaso0} shows a small network.
\end{frame}

\begin{frame}
    \frametitle{Figures}
    \begin{table}
        \begin{tabular}{rccc}
            \toprule
            & Competitor 1 & Competitor 2 & we\\
            \midrule
            Feature A & \no & \maybe & \yes\\
            Feature B & \no & \maybe & \yes\\
            Feature C & \no & \maybe & \yes\\
            Feature D & \no & \maybe & \yes\\
            \bottomrule
        \end{tabular}
    \end{table}
\end{frame}


% Include markdown source from ./pandoc
%\section{Section heading}

\begin{frame}
    \frametitle{Example frame}
    \begin{itemize}
        \item item 1
        \item $\ldots$
        \begin{itemize}
            \item test
            \item $\ldots$
        \end{itemize}
    \end{itemize}
    Citation \cite{rfc959}\footnote{A footnote}

    \paragraph{Math mode should be fully functional:}
    $$
    \hat s
    \overline s
    \mathcal S
    \mathbit S
    \mathbit \Lambda
    \sum
    \pd{\xi}
    \pr{X=0}
    \mathbit 1
    $$
\end{frame}

\begin{frame}
    \frametitle{Figures}
    \begin{figure}
        \centering
        \includegraphics[width=.5\textwidth]{figures/example}
        \caption{Figure caption}
        \label{Maizaso0}
    \end{figure}
    Figure~\ref{Maizaso0} shows a small network.
\end{frame}

\begin{frame}
    \frametitle{Figures}
    \begin{table}
        \begin{tabular}{rccc}
            \toprule
            & Competitor 1 & Competitor 2 & we\\
            \midrule
            Feature A & \no & \maybe & \yes\\
            Feature B & \no & \maybe & \yes\\
            Feature C & \no & \maybe & \yes\\
            Feature D & \no & \maybe & \yes\\
            \bottomrule
        \end{tabular}
    \end{table}
\end{frame}


% Comment out if you do not want a bibliography
\section{Bibliography}
\begin{frame}[allowframebreaks]
    \bibliographystyle{abbrv}
    \setbeamertemplate{bibliography item}[text]
    \footnotesize
    \bibliography{lit}
\end{frame}

\end{document}



\begin{document}

% If you are preparing a talk but do not like the default font sizes, you may
% want to try the class option 'beameralt', which uses smaller default font
% sizes and integrates subsection/subsubsection names into the headline.

% For lecture mode, you may want to build one set of slides per chapter but
% with common page numbering. If so,
% 1) create a new .tex file for each chapter, e.g. slides_chapN.tex,
% 2) set the part counter to N-1 (assuming chapters start at 0), and
% 3) and name your chapter by using the \part{} command.
%\setcounter{part}{-1}
%\part{Organisatorisches und Einleitung}

% For 16:9 slides, use the class option 'aspectratio=169'.

% If class option 'noframenumbers' is given, frame numbers are not printed.

% If class option 'notitleframe' is given, the title frame is not autmatically
% generated.

% Class option 'nocontentframes' suppresses automatic generation of content
% frames when new parts/sections are started.

% Include source files from ./include (or ./include/chapN).
\section{Section heading}

\begin{frame}
    \frametitle{Example frame}
    \begin{itemize}
        \item item 1
        \item $\ldots$
        \begin{itemize}
            \item test
            \item $\ldots$
        \end{itemize}
    \end{itemize}
    Citation \cite{rfc959}\footnote{A footnote}

    \paragraph{Math mode should be fully functional:}
    $$
    \hat s
    \overline s
    \mathcal S
    \mathbit S
    \mathbit \Lambda
    \sum
    \pd{\xi}
    \pr{X=0}
    \mathbit 1
    $$
\end{frame}

\begin{frame}
    \frametitle{Figures}
    \begin{figure}
        \centering
        \includegraphics[width=.5\textwidth]{figures/example}
        \caption{Figure caption}
        \label{Maizaso0}
    \end{figure}
    Figure~\ref{Maizaso0} shows a small network.
\end{frame}

\begin{frame}
    \frametitle{Figures}
    \begin{table}
        \begin{tabular}{rccc}
            \toprule
            & Competitor 1 & Competitor 2 & we\\
            \midrule
            Feature A & \no & \maybe & \yes\\
            Feature B & \no & \maybe & \yes\\
            Feature C & \no & \maybe & \yes\\
            Feature D & \no & \maybe & \yes\\
            \bottomrule
        \end{tabular}
    \end{table}
\end{frame}


% Include markdown source from ./pandoc
%\section{Section heading}

\begin{frame}
    \frametitle{Example frame}
    \begin{itemize}
        \item item 1
        \item $\ldots$
        \begin{itemize}
            \item test
            \item $\ldots$
        \end{itemize}
    \end{itemize}
    Citation \cite{rfc959}\footnote{A footnote}

    \paragraph{Math mode should be fully functional:}
    $$
    \hat s
    \overline s
    \mathcal S
    \mathbit S
    \mathbit \Lambda
    \sum
    \pd{\xi}
    \pr{X=0}
    \mathbit 1
    $$
\end{frame}

\begin{frame}
    \frametitle{Figures}
    \begin{figure}
        \centering
        \includegraphics[width=.5\textwidth]{figures/example}
        \caption{Figure caption}
        \label{Maizaso0}
    \end{figure}
    Figure~\ref{Maizaso0} shows a small network.
\end{frame}

\begin{frame}
    \frametitle{Figures}
    \begin{table}
        \begin{tabular}{rccc}
            \toprule
            & Competitor 1 & Competitor 2 & we\\
            \midrule
            Feature A & \no & \maybe & \yes\\
            Feature B & \no & \maybe & \yes\\
            Feature C & \no & \maybe & \yes\\
            Feature D & \no & \maybe & \yes\\
            \bottomrule
        \end{tabular}
    \end{table}
\end{frame}


% Comment out if you do not want a bibliography
\section{Bibliography}
\begin{frame}[allowframebreaks]
    \bibliographystyle{abbrv}
    \setbeamertemplate{bibliography item}[text]
    \footnotesize
    \bibliography{lit}
\end{frame}

\end{document}



\begin{document}

% If you are preparing a talk but do not like the default font sizes, you may
% want to try the class option 'beameralt', which uses smaller default font
% sizes and integrates subsection/subsubsection names into the headline.

% For lecture mode, you may want to build one set of slides per chapter but
% with common page numbering. If so,
% 1) create a new .tex file for each chapter, e.g. slides_chapN.tex,
% 2) set the part counter to N-1 (assuming chapters start at 0), and
% 3) and name your chapter by using the \part{} command.
%\setcounter{part}{-1}
%\part{Organisatorisches und Einleitung}

% For 16:9 slides, use the class option 'aspectratio=169'.

% If class option 'noframenumbers' is given, frame numbers are not printed.

% If class option 'notitleframe' is given, the title frame is not autmatically
% generated.

% Class option 'nocontentframes' suppresses automatic generation of content
% frames when new parts/sections are started.

% Include source files from ./include (or ./include/chapN).
\section{Section heading}

\begin{frame}
    \frametitle{Example frame}
    \begin{itemize}
        \item item 1
        \item $\ldots$
        \begin{itemize}
            \item test
            \item $\ldots$
        \end{itemize}
    \end{itemize}
    Citation \cite{rfc959}\footnote{A footnote}

    \paragraph{Math mode should be fully functional:}
    $$
    \hat s
    \overline s
    \mathcal S
    \mathbit S
    \mathbit \Lambda
    \sum
    \pd{\xi}
    \pr{X=0}
    \mathbit 1
    $$
\end{frame}

\begin{frame}
    \frametitle{Figures}
    \begin{figure}
        \centering
        \includegraphics[width=.5\textwidth]{figures/example}
        \caption{Figure caption}
        \label{Maizaso0}
    \end{figure}
    Figure~\ref{Maizaso0} shows a small network.
\end{frame}

\begin{frame}
    \frametitle{Figures}
    \begin{table}
        \begin{tabular}{rccc}
            \toprule
            & Competitor 1 & Competitor 2 & we\\
            \midrule
            Feature A & \no & \maybe & \yes\\
            Feature B & \no & \maybe & \yes\\
            Feature C & \no & \maybe & \yes\\
            Feature D & \no & \maybe & \yes\\
            \bottomrule
        \end{tabular}
    \end{table}
\end{frame}


% Include markdown source from ./pandoc
%\section{Section heading}

\begin{frame}
    \frametitle{Example frame}
    \begin{itemize}
        \item item 1
        \item $\ldots$
        \begin{itemize}
            \item test
            \item $\ldots$
        \end{itemize}
    \end{itemize}
    Citation \cite{rfc959}\footnote{A footnote}

    \paragraph{Math mode should be fully functional:}
    $$
    \hat s
    \overline s
    \mathcal S
    \mathbit S
    \mathbit \Lambda
    \sum
    \pd{\xi}
    \pr{X=0}
    \mathbit 1
    $$
\end{frame}

\begin{frame}
    \frametitle{Figures}
    \begin{figure}
        \centering
        \includegraphics[width=.5\textwidth]{figures/example}
        \caption{Figure caption}
        \label{Maizaso0}
    \end{figure}
    Figure~\ref{Maizaso0} shows a small network.
\end{frame}

\begin{frame}
    \frametitle{Figures}
    \begin{table}
        \begin{tabular}{rccc}
            \toprule
            & Competitor 1 & Competitor 2 & we\\
            \midrule
            Feature A & \no & \maybe & \yes\\
            Feature B & \no & \maybe & \yes\\
            Feature C & \no & \maybe & \yes\\
            Feature D & \no & \maybe & \yes\\
            \bottomrule
        \end{tabular}
    \end{table}
\end{frame}


% Comment out if you do not want a bibliography
\section{Bibliography}
\begin{frame}[allowframebreaks]
    \bibliographystyle{abbrv}
    \setbeamertemplate{bibliography item}[text]
    \footnotesize
    \bibliography{lit}
\end{frame}

\end{document}

