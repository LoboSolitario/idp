
% Alle Konfigurationsbefehle sind optional. Fehlende Befehle fueheren einfach
% zu "blank forms".

% Type of thesis
% valid parameters: bachelor,master,diplom,idp,gr,hiwi,other
% If 'other' is chosen an optional parameter can be passed along (\type[optional]{other}).
% Without an optional parameter 'other' is chosen as a description.
\type{bachelor}

% \studiengang{} details the field of study
% valid parameters:
%   "Informatik",
%   "Wirtschaftsinformatik",
%   "Robotics, Cognition, Intelligence"
%   "Informatik: Games Engineering"
\studiengang{Informatik}

% Informationen ueber den Studenten. Sollte selbsterklaerend sein.
\anrede{Herr}
\nachname{nachname}
\vorname{vorname}
\matrikel{matrikel}
\rbgaccount{rbgaccount}
\semester{1}{SoSe\,2016}
\studientelefon{}{tel}
\heimattelefon{}{--}
\studienadresse{strasse}{plz stadt}
\heimatadresse[adresszusatz=,appartment=]{}{}
\mail{student@tum.de}

% Informationen ueber die Arbeit. Sollte selbsterklaerend sein.
\themensteller{\NEThead}
\beginn{04}{2016}
\endt{08}{2016}
\betreuer{Nyan Cat, Grumpy Cat}
\title{English Title of My Thesis}{Englischer Titel Meiner Arbeit}

% Nur für Bachelorarbeiten
\sprache{englisch} % options: deutsch | englisch

% Falls \type{hiwi} gesetzt wurde, wird die Taetigkeit auf dem Aufnahmeformular
% des Lehrstuhls angegeben.
\taetigkeit{test}

