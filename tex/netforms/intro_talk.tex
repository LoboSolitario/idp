% intro talk in german
%\documentclass[NET,a4paper,12pt,ngerman]{netforms}

% intro talk in english
\documentclass[NET,a4paper,12pt,english]{netforms}

\usepackage[utf8]{inputenc}
\usepackage{tumlang}
\usepackage{tumcontact}
\usepackage{scrlayer-scrpage}
\usepackage[textsize=scriptsize]{todonotes}
% Disable todonotes with
%\setuptodonotes{disable}
\setuptodonotes{inline}
\usepackage{pgfgantt}
\usepackage{float}
\usepackage{enumitem}
\usepackage{url}
\newlist{researchquestions}{enumerate}{1}
\setlist[researchquestions,1]{label=\textbf{\textit{RQ \arabic*}}, resume}
\newlist{goals}{enumerate}{1}
\setlist[goals,1]{label=\textbf{\textit{Goal \arabic*}}, resume}


\geometry{%
	top=20mm,
	bottom=20mm,
	left=25mm,
	right=25mm,
	headsep=1.5cm,
	includehead,
}

% Alle Konfigurationsbefehle sind optional. Fehlende Befehle fueheren einfach
% zu "blank forms".

% Type of thesis
% valid parameters: bachelor,master,diplom,idp,gr,hiwi,other
% If 'other' is chosen an optional parameter can be passed along (\type[optional]{other}).
% Without an optional parameter 'other' is chosen as a description.
\type{bachelor}

% \studiengang{} details the field of study
% valid parameters:
%   "Informatik",
%   "Wirtschaftsinformatik",
%   "Robotics, Cognition, Intelligence"
%   "Informatik: Games Engineering"
\studiengang{Informatik}

% Informationen ueber den Studenten. Sollte selbsterklaerend sein.
\anrede{Herr}
\nachname{nachname}
\vorname{vorname}
\matrikel{matrikel}
\rbgaccount{rbgaccount}
\semester{1}{SoSe\,2016}
\studientelefon{}{tel}
\heimattelefon{}{--}
\studienadresse{strasse}{plz stadt}
\heimatadresse[adresszusatz=,appartment=]{}{}
\mail{student@tum.de}

% Informationen ueber die Arbeit. Sollte selbsterklaerend sein.
\themensteller{\NEThead}
\beginn{04}{2016}
\endt{08}{2016}
\betreuer{Nyan Cat, Grumpy Cat}
\title{English Title of My Thesis}{Englischer Titel Meiner Arbeit}

% Nur für Bachelorarbeiten
\sprache{englisch} % options: deutsch | englisch

% Falls \type{hiwi} gesetzt wurde, wird die Taetigkeit auf dem Aufnahmeformular
% des Lehrstuhls angegeben.
\taetigkeit{test}



\pagestyle{scrheadings}
\clearscrheadfoot
\chead{\TUMheader{1cm}}

\renewcommand{\maketitle}{%
	\begin{center}
		\textbf{\introductoryheadline}%

		\Large%
		\textbf{\thetitle}%
	\end{center}

	\footnotesize%
	\hrule
	\vskip1ex
	\begin{tabular}{ll}
		\thenamelabel: & Parshant Singh, Shouvik Ghosh\\
		\theadvisorlabel: & \hspace*{-.5ex}\thebetreuer\\
		\thesupervisorlabel: & \chairhead\\
		\thebeginlabel: & \thebeginnmonat/\thebeginnjahr\\
		\theendlabel: & \theendmonat/\theendjahr\\
	\end{tabular}
	\vskip1ex
	\hrule
	\vskip4ex
}

\linespread{1.2}
\setlength{\parskip}{.5\baselineskip}

\begin{document}
\maketitle

\subsection*{Topic}

% \todo{Add Motivation: Why is this topic relevant? Which ``big'' problem does it solve?}
 With the growing adoption of blockchain technology, the number of readily-available solutions has multiplied dramatically. Approximately five thousand distinct cryptocurrencies have been reported on a single website \cite{1}. Each of these implementations aims at offering improvements through distinctive features, focused on the performance and application to various use cases. The authors of different blockchain protocols claim impressive performance. These results are usually obtained in isolation and are often non-reproducible, which makes them hard to compare and verify. Existing blockchain benchmark frameworks are either specifically focused on testing a single protocol or do not support or provide workloads that reflect the real-world usage of the blockchain system. Diablo (DIstributed Analytical Blockchain benchmark framework) is a prototype benchmark framework for blockchain protocols initially developed by Chris Natoli in 2021 \cite{2}. As of Oct 2022, as part of previous work \cite{3}, 6 blockchain protocols (Algorand, Avalanche, Diem, Quorum, Ethereum, Solana) have already been evaluated using the Diablo framework. In previous work at the Chair \cite{3}, the Diablo framework was used to evaluate all of the previous protocols, without their profiling. In this work, we aim to extend the functionality of the Diablo framework by two additional popular blockchain protocols - Near \cite{4} and Dfinity \cite{5} and continue with a detailed analysis of two of the currently supported protocols.  

\subsection*{Goals}

\begin{goals} [leftmargin=.5in]
\item Singh, Parshant - Extend the Diablo Framework by Near.\\
Ghosh, Shouvik - Extend the Diablo Framework by Dfinity.
\item Detailed experiments to identify possible bottlenecks and look for optimization techniques for blockchains Algorand \cite{6} and possibly Solana \cite{7} or Avalanche \cite{8}.\\
Singh, Parshant - Algorand \\ 
Ghosh, Shouvik - Solana or Avalanche
\end{goals}

\subsection*{Approach}
\subsubsection*{Goal 1 (Applies to both Parshant Singh and Shouvik Ghosh) }
We will need to get familiar with the Diablo framework \cite{9} and the individual protocols. In the beginning, we will focus on understanding how the Diablo framework is deployed and can be used for benchmarking framework. After that, we will be deploying the blockchain validator nodes (NEAR and Dfinity) in our testbed. Once that is complete we will be using the diablo framework to create transactions and benchmark the blockchain network. At the end, we will conduct experiments verifying the implementation works and assess various parameters such as a number of nodes, stake distribution, and transaction latency/finality. As outlined previously, each team member will focus on the individual Blockchain protocols.

\subsubsection*{Goal 2 (Applies to both Parshant Singh and Shouvik Ghosh)}
The second goal focuses on the performance evaluation of blockchain protocols called Algorand and Solana or Avalanche using Diablo for transaction generation and chair testbed for detailed analysis. The framework and the performance evaluation would require some updates/extensions and a more detailed evaluation of the protocols. Our task will be to get familiar with the Algorand, Solana or Avalanche protocols, Diablo framework, the chair testbed, and using the knowledge from Goal 1 about Diablo framework and chair testbed, assess and identify certain performance bottlenecks and possibly their improvements, e.g, on the network layer (optimize gossip) or processing layer (cryptographic operation). Work distribution will be as defined in the goals section.

\subsection*{Justification of the lecture (Data Networking - EI70330)}
A strong background in networking is essential to understand the underlying concepts in Blockchain. We need to have a thorough understanding of ISO/OSI models and relevant internet metrics to assess Blockchain performance correctly. This course dives deeper into the communication networks and their building blocks which is certainly relevant to our main area of focus in this IDP. We will gain a good understanding of the principles of resource management for wireless and wireline packet-based communication and apply these for resource management, and traffic engineering of network protocols and architectures. This understanding will be helpful while setting up communication between different machines on the testbed for the blockchain nodes' and the Diablo Framework interaction.


\subsection*{Planned Schedule (Applies to both Parshant Singh and Shouvik Ghosh)}
\begin{figure}[H]
  \centering
  \begin{ganttchart}[
      x unit=0.81mm,
      y unit title=0.6cm,
      title height=1,
      title label font=\scriptsize,
      bar label font=\scriptsize,
      milestone label font=\scriptsize,
      milestone inline label node/.append style={left=2mm},
      milestone/.append style={fill=orange, shape=rectangle},
      y unit chart=.6cm,
      time slot format=isodate,
      time slot unit=day,
      inline,
      bar height=0.7,
    ]{2022-11-01}{2023-05-15}
    \gantttitlecalendar{month=name}\\
    \ganttbar{Topic Familiarization}{2022-11-01}{2022-11-30}\\
    \ganttbar{Literature Research}{2022-11-22}{2022-12-31}\\
    \ganttbar{Diablo Framework familiarity}{2022-11-05}{2023-02-1}\\
    \ganttbar{Profiling of a blockchain protocol}{2023-02-01}{2023-03-31}\\
    \ganttbar{Identify possible improvements}{2023-03-01}{2023-04-30}\\
    \ganttbar{Writing of a report}{2023-01-01}{2023-05-15}\\
    % \ganttmilestone{Intermediate talk}{2016-06-15}\\
    % \ganttmilestone{Submission}{2016-08-14}
  \end{ganttchart}
  \label{fig:work-plan}
\end{figure}

\bibliographystyle{IEEEtran}
\scriptsize
\bibliography{IEEEabrv,lit}

\end{document}

